%----------------------------------------------------------------------------------------
%   PACKAGES AND DOCUMENT CONFIGURATIONS
%----------------------------------------------------------------------------------------

\documentclass[11pt]{article}
	\usepackage{geometry} % Pour passer au format A4
	\geometry{hmargin=1cm, vmargin=1cm} %
	
	\usepackage{graphicx} % Required for including pictures
	\usepackage{float} %
	\usepackage{pdfpages}
	
	%Français
	\usepackage[T1]{fontenc}
	\usepackage[english,francais]{babel}
	\usepackage[utf8]{inputenc}
	
	\usepackage{eurosym,siunitx,lmodern,url,hyperref}
	\usepackage{multicol,multido}
	
	%Maths
	\usepackage{amsmath,amsfonts,amssymb,amsthm,gensymb}
	
	%Autres
	\linespread{1} % Line spacing
	\setlength\parindent{0pt} % Removes all indentation from paragraphs
	
	\renewcommand{\labelenumi}{\alph{enumi}.} %
	
	\pagestyle{empty}
	
	% Droite horizontal avec argument d'épaisseur
	\newcommand{\horrule}[1]{\rule{\linewidth}{#1}}
	% Pointillé avec argument de nombre de lignes
	\newcommand{\Pointille}[1][3]{\multido{}{#1}{ \makebox[\linewidth]{\dotfill}\\[\parskip]}}
	
	%-------------------------------------------------------------------------------
	%   DOCUMENT INFORMATION
	%-------------------------------------------------------------------------------
	
	\begin{document}
	
	\begin{titlepage}
	  
	      \center
	      \horrule{2px} \\[2.5cm]
	      \textsc{\LARGE Trame d'expérience}\\[1.5cm] %
	      \textsc{\Large Compétences / Relaxation}\\[2.5cm]
	          \vfill % Rempli le reste de la page avec du blanc
	  
	\end{titlepage}
	
	\tableofcontents % Include a table of contents
	\newpage % Première section sur une nouvelle page.
	
	\section{Indexation}
	Compétences, relaxation, climat scolaire.
	
	\newpage
	
	\section{Constats de départ / Diagnostic}
	
	Les relations élèves / élèves, élèves / adultes sont parfois tendus et entraîne une dégradation du climat scolaire. 
	
	Lorsqu'un adulte crie, cela peut produire l'effet inverse à celui souhaité en aggravant la situation et en fermant toutes possibilités de dialogue. Il n'aime pas qu'on leur crie dessus.
	
	Les élèves ont de longues journées. Les adultes aussi. Beaucoup de temps est occupé pour la transmission de connaissance. Cela entraîne de la fatigue. Les élèves n'ayant finalement que peu de temps de repos dans une journée.
	
	Les classes sont parfois bruyantes. Dès lors, il devient compliqué pour un élève de poser une question et d'entendre la réponse.
	
	\newpage
	
	\section{Objectif visé / Hypothèse de départ}
	
	L'objectif premier est d'apaiser le climat scolaire. 
	
	La relaxation peut permettre d'améliorer la communication inter partie et participe à l’acquisition de cette compétence.
	
	\newpage
	
	\section{Descriptif détaillé de l'expérience}
	
	\subsection{Qui ? / Pour qui ?}
	
	Les professeurs pour les élèves de cycle 3 et cycle 4. Ainsi que pour soit-même. 
	
	En effet un apaisement du climat scolaire est bénéfique à la fois au élèves mais aussi aux professeurs. 
	
	Il y a beaucoup de rétro-action (feedback) dans la communication. Une amélioration de l'humeur des élèves entraîne une amélioration de l'humeur du professeur qui entraîne une amélioration de l'humeur des élèves.
	
	Classe de Sixième : 
	Classe hétérogène. Les codes du collège ne sont pas acquis : Lever la main pour parler, ne pas se déplacer sans autorisation, ne pas stresser lors des changements de salle en fin de cours. 
	
	Classe de Quatrième :
	4a : Classe hétérogène, grosse tête de classe autonome. Petite groupe avec des difficultés. Petit groupe avec des lacunes très importantes.
	
	4b : Classe homogène mais faible. Très peu confiance en elle. Beaucoup de bavard.
	
	Classe de Troisième
	Petit groupe car en ré-évaluation.
	
	\subsection{Comment ?}
	
	La mise en place d'une musique douce à des moments stratégiques (Quand ?) peut aider à apaiser le climat de classe. Le professeur peut alors se taire et demander un silence relatif. La musique attire alors certains élèves qui vont avoir tendance sans s'en rendre compte à l'écouter et ne plus trop réfléchir au reste.
	
	En fonction du quand ?
	
	\subsubsection{La fin du cours}
	
	5 minutes avant la fin du cours. Cela peut être accompagné de mot de relaxation pour leur souhaiter une bonne journée, un bon cours suivant. On peut leur proposer de profiter de ce moment pour se détendre, ne penser à rien et ne faire qu'uniquement écouter. 
	
	La musique avec son niveau sonore constant permet de se positionner face au bruit. Si je n’entends plus la musique, c'est que je parle trop fort.
	
	Ceux qui n'ont pas terminé leurs petites affaires le font. Tandis que les autres n'ont pas, rien à faire mais ont quand même une tâche possible pour s'occuper. Écouter la musique. Pour autant, cette tâche ne les sollicite pas trop sur le plan des connaissances.
	
	La classe de sixième a été la principale cible de mon expérience. 
	
	Il devient plus facile de communiquer une fois le calme relativement revenu. 
	
	\subsubsection{Les évaluations}
	
	Pendant les évaluations, le but est de proposer un fond sonore neutre qui permet de couvrir légèrement les petits bruits de ça et là présent en classe. Il est aisé de se servir de rapport temporel puisque le fond sonore est un enchaînement de musique. 
	
	Quelques élèves ressentent le besoin de poser des questions. Parfois, la présence d'un fond sonore peut suffire à les dissuader de poser leur question.
	
	Avec moins de question posé, il devient plus aisé de communiquer avec les élèves dans le besoin.
	
	La présence d'un professeur qui surveille avec une posture inquisitrice peut amener du stress auprès de l'élève. Le professeur montre également une autre facette de lui. Il n'est plus dans l'attente et dans la surveillance, mais se montre aussi dans l'écoute.
	
	\subsection{Quoi ?}
	
	Morceaux : 
	
	\begin{itemize}
	\item \url{https://www.youtube.com/watch?v=LVqtGejYe_o}
	\item \url{https://www.youtube.com/watch?v=_Jkla2DZu5g}
	\end{itemize}
	
	La principale compétence travaillée est communiquer. 
	Soit par une absence de parole, soit grace à un langage plus adapté.
	
	J'ai choisi d'évaluer principalement lors des évaluations en demandant de répondre à un sondage à choix multiples et bulletin fermé.
	Le bulletin fermé a été choisi après un premier essai à bulletin ouvert où l’effet de groupe a semblé avoir une influence non négligeable. 
	
	Les bulletins fermés sont quand à eux plus pénibles pour les élèves naturellement curieux et désireux de se fondre dans la masse et souhaitant votre "comme le copain".
	
	J'ai volontairement respecter l'anonymat pour le sondage. Pour autant, il est assez simple de se faire une idée assez précise du vote de la plupart des élèves de la classe.
	
	Les choix possibles sont : 
	
	\begin{itemize}
	\item La musique m'a énervé. 
	      \item Je n'en vois pas l'intérêt.
	      \item La musique ne m'a pas gené.
	      \item La musique m'a apaisé
	\end{itemize}
	
	Sur ces séances, j'ai également évalué subjectivement l'état de la classe.
	
	\begin{itemize}
	\item Indifférent
	      \item Agacé
	      \item Apaisé
	\end{itemize}
	
	\subsection{Combien ?}
	
	Cette expérience a été mené par un professeur dans des classes de tailles restreintes avec des groupes de 20 à 27 élèves.
	
	\subsection{Où ?}
	
	L'expérience a été mené en salle de classe.
	
	\subsection{Quand ?}
	
	L'experience a porté sur deux moments anxiogènes. 
	
	\begin{itemize}
	\item La fin du cours.  
	\item Les évaluations.
	\end{itemize}
	
	\subsubsection{La fin du cours}
	
	La fin des cours est un moment anxiogène. Les élèves veulent sortir à l'heure par peur d'être en retard au le cours prochain, par peur de loupé de la récréation, temps de pause salvateur pour souffler. Ils doivent souvent noté les devoirs parfois transmis à l'oral. Ils doivent faire leur sac. Ils doivent rester assis. Ils doivent attendre la sonnerie...
	
	Cela a été proposé à des moments particulièrement tendus pour apaiser. Cela a été également proposé à des moments où l'apaisement était déjà présent pour instaurer et associer du calme à ses petits moments.
	
	\subsubsection{Les évaluations}
	
	Les évaluations sont des moments anxiogènes. Les élèves en difficultés comme les autres sont stressés par les notes et ont souvent besoin d'être rassuré en posant des questions. La principale question étant "Ai-je juste ?". D'autre élèves ont juste besoin de calme et de silence afin de pouvoir se concentrer et se lancer.
	
	Lorsqu'une classe pose trop de questions, cela a tendance à avoir un effet boule de neige. 
	
	\newpage
	
	\section{Evaluation de l'expérience}
	
	\subsection{Effets, critères et indicateurs}
	
	\subsubsection{critères d'évaluations}
	
	L'état émotionnel d'une personne est un complexe à juger et évaluer sans outils précis. Pour autant, il semble possible d'une part de percevoir les modifications pour un groupe de 20/30 élèves. 
	
	Les évaluations sont restées par classe pour essayer de dégager un profil dans l'optique d'adapter aux besoins.
	
	\subsubsection{Indicateur d'évaluations}
	
	Les sondages ont été dépiautés. Une brève analyse a été faite. Principalement visuellement. En effet, les échantillons sont petits (moins de 30) et redondant (concentrer sur 4 classes). Les caractères étant qualitatifs, peu d'outils sont à ma disposition pour en faire une étude plus poussée.
	
	\subsection{Analyse}
	
	L'analyse des données est sans doute la partie la plus intéressante. Les classes ont réagit dans l'ensemble assez différemment et je ne possède pas assez de données (classes différentes) pour pouvoir en regrouper et formuler des profils.
	
	\subsection{6 - fin de cours}
	Les objectifs ont été atteints.
	
	Le groupe classe a compris que lorsque je mettais la musique douce, le cours était fini et que mon seul objectif était à présent de chercher à les apaiser pour la suite. 
	
	Ils ont compris que j'étais en attente de pouvoir entendre la musique. Cela me conforte dans l'idée que voir un professeur occupé les apaisent plus que de voir un professeur en attente mais inactif. 
	
	Parfois un peu d'incompréhension si l'expérience est menée en dernière heure de la journée surtout le vendredi. Ils sont particulièrement épuisés et sur les nerfs, mais certains élèves veulent garder cette état pour se défouler une fois en dehors du collège.
	
	Quelques questions sur les premières séances : c'est quoi, pourquoi. Les réponses ont toujours été données.
	
	Parfois un peu d'attente : M. On se met la musique douce. Mais cela n'arrive qu'en fin de séance déjà calme.
	
	Si la musique est lancée trop tard ou avec trop d'élève occupé, alors c'est un échec.
	
	\subsection{4a - en éval, 3 - ré-éval}
	Les objectifs ont été atteints.
	
	$$\dfrac{2}{3}$$ des élèves : La musique ne m'a pas gené ou La musique m'a apaisé.
	1 ou 2 élèves : génés.
	5 ou 6 élèves : indifférents
	
	Le groupe est déjà assez autonome. La musique permet de couvrir les micro bavardages. Les élèves avec des lacunes monstres apprécient aussi car ça les occupent.
	Les élèves en difficultés sont parfois indifférents, mais remarque que je suis disponible ce qui est intéressant.
	
	Pour le groupe troisième, la musique signifie ré-évaluation et concerne donc déjà des élèves en attente d'un bon climat de travail.
	
	\subsection{4b - en éval}
	Les objectifs n'ont pas été atteints.
	
	$$\dfrac{1}{3}$$ des élèves : La musique ne m'a pas gêné ou La musique m'a apaisé.
	Le reste de géné
	
	Le groupe n'est pas autonome et à besoin d'un accès constant au professeur pour poser des questions, demander de l'aide et valider des réponses. Tout le monde veux communiquer avec tout le monde...
	
	Le refus de répondre à ces trop nombreuses sollicitations de la part du professeur a créer un agacement auprès des élèves. 
	
	\newpage
	
	\section{Perspectives}
	
	L'un des avantages de cette méthode est la facilité de mise en place. L'investissement est ici mineur. Il n'y a rien à connaître si ce n'est un pc connecté. En effet, le professeur, ni les élèves n'animent. 
	
	Il faut recueillir plus de données. 
	
	La relaxation en 4b n'est pas perdu pour autant, il faut trouver un autre moyen de les relaxer. D'autre méthodes de relaxation sont possible sur d'autre temps. 
	
	\end{document}